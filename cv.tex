\documentclass[12pt, a4paper, sans]{moderncv}

\moderncvstyle{classic}                       
\moderncvcolor{red} 

\usepackage[scale=0.9]{geometry}

\usepackage[unicode]{hyperref}
\definecolor{linkcolour}{rgb}{0,0.2,0.6}
\hypersetup{colorlinks,breaklinks,urlcolor=linkcolour, linkcolor=linkcolour}

\name{Evgeniy}{Kirpichenko}
\address{Kyiv, Ukraine}
\mobile{+380~66~976~1745} 
\email{evkirpichenko@gmail.com}
\social[github]{kirpichenko}
\extrainfo{Skype: jenek.ukr}

\begin{document}

\makecvtitle

\section{Work Experience}
\cventry
	{Jul 2014 -- Present}
	{iOS Developer}
	{\href{http://omnigon.com}{Omnigon Communication LLC}}
	{Kyiv, Ukraine}{}
	{Working on official Verizon IndyCar series iOS application.}

\cventry
	{Jul 2010 -- Jun 2014}
	{iOS Developer}
	{\href{http://mlsdev.com}{MLSDev LLC}}
	{Donetsk, Ukraine}{}
	{Assessment of requirements for software development and time estimation. 
	\newline Software development.}
	
\section{Commercial Projects}
\cvitem {
	\href{https://itunes.apple.com/us/app/indycar-15/id606905722?mt=8}{IndyCar 15}}{
	The Official Verizon IndyCar Series Mobile application. Racing news and media are available in the app even when your device is offline. View leaderboard, car telemetry and other real-time features during live races.}
		
\cvitem {
	\href{https://itunes.apple.com/us/app/brain-power-free-gangnam-style/id600235919?mt=8}{Brain Power}}{
	Classic game to test your memory. Memorize the sequence played by the game and repeat it. The faster you tap the sequence, the higher is your scores.}
	 
\cvitem {
	\href{https://itunes.apple.com/us/app/muddler/id498512664?mt=8}{Muddler}}{
	With Muddler you can discover, upload, rate and share the best drinks in cities around the world.}
	
\section{Open-source Projects}
\cvitem {
	\href{https://github.com/kirpichenko/EKKeyboardAvoiding}{Keyboard Avoiding}}{
	It's an universal solution for keyboard avoiding for iOS that allows to view the content placed under the presented keyboard.}

\cvitem{
	\href{https://github.com/kirpichenko/EKTilesMaker}{Tiles Maker}}{
	A simple iOS utility that slices provided original image onto tiles. Tiles may be used for displaying large image on different detalization levels.}
	
\section{Technical skills}
\cvitem {Language}{
	Objective C}

\cvitem {Frameworks and Tools}{
	AFNetworking, Google Protobuf, Bonjour, Sockets, MagicalRecord, CocoaPods, Cedar, XCTest, Calabash-iOS}

\cvitem {Process}{
	TDD, BDD, XP, Continuous integration, Design patterns}

\cvitem {VCS}{
	Git, Mercurial}

\cvitem {CRM}{
	Jira, Redmine}
		
\section{Languages}
\cvitem {Native}{
	Russian, Ukrainian}

\cvitem {Advanced}{
	English}

			
\section{Education}
\cventry
	{2005 -- 2010}
	{Master of Applied Mathematics}
	{Donetsk National University}
	{Donetsk, Ukraine}{}
	{Faculty of Mathematics}

\end{document}
